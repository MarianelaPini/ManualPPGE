\chapter{Bem-vindo(a)}
%\addcontentsline{toc}{chapter}{Apresentação}
\label{chap:aprest}

Parabéns a você que acaba de ingressar no Programa de Pós-Graduação em
Ecologia da Universidade de São Paulo (PPGE-USP). Toda a equipe do
PPGE-USP está empenhada em oferecer um ambiente intelectual
estimulante a seus estudantes, com muitas oportunidades de formação e
crescimento profissional. Esta cartilha tem as primeiras orientações
para que você possa melhor aproveitar essas oportunidades. Para isso,
nos preocupamos em apresentar de forma resumida e didática o
funcionamento de nosso Programa, especialmente quanto aos aspectos
mais importantes para as alunas e os alunos ingressantes.

Portanto, \textbf{este documento não contém nem revê exaustivamente
todas as normas e procedimentos de nosso Programa ou da Pós- Graduação
da USP}. Você deve conhecer as normas consultando as seguintes páginas
na internet:

\begin{description}
\item[PPGE-USP:] \url{http://www.posecologia.ib.usp.br}
\item[Comissão de Pós-Graduação do IBUSP:] \url{http://pos.ib.usp.br/}
\item[Pró-Reitoria de Pós-graduação USP:] \url{http://www.usp.br/prpg/}
\end{description}


\section{Nosso  programa}

O objetivo geral do Programa de Pós-Graduação em Ecologia da USP
(PPGE-USP) é formar mestres e doutores em ecologia com amplo
entendimento de todos os níveis de organização ecológica, que
contribuam para o progresso intelectual da área de biodiversidade e
para a resolução de problemas ecológicos e ambientais relevantes, com
entendimento crítico da atividade científica e seu papel na sociedade.
Para tanto, buscamos oferecer um ambiente estimulante de aprendizado
em:

\begin{enumerate}
\item Formulação de perguntas e hipóteses de relevante interesse na área;
\item Sólida formação teórica e analítica;
\item Habilidade de comunicação seja por meio escrito ou falado, para
  público acadêmico e não-acadêmico;
\item Autonomia intelectual, rigor e senso crítico para a análise,
  interpretação e aplicação dos achados da pesquisa para a resolução
  de questões ecológicas acadêmicas e aplicadas;
\item Aproximação da pesquisa com a prática, com compromisso social.
\end{enumerate}

Adicionalmente, proporcionamos condições para o debate e a divulgação
dos trabalhos científicos, tanto em âmbito nacional como internacional.
Nosso mestrado foi criado em 1982 e o doutorado em 1993. Até o final de
2020, formamos 451 profissionais, sendo que 343 cursaram o mestrado, 190
cursaram o doutorado e 82 cursaram ambos e hoje atuam em diversas áreas,
tanto no meio acadêmico como em órgãos governamentais, setor privado e
organizações não governamentais.

O PPGE-USP tem orientadores e orientadoras em três subáreas: i)
Ecologia aplicada (conservação, restauração e educação); ii) Ecologia
de Ecossistemas, de Populações e de Comunidades; iii) Ecologia
Evolutiva e Comportamental.  Em nossa página na internet você
encontrará a lista dos orientadores credenciados, \emph{links} para
suas páginas e currículos, e informações para contato.


\section{A Comissão Coordenadora do Programa}
\label{sec:CCP}

\textbf{A Comissão Coordenadora do Programa (CCP) é a instância
colegiada que coordena, normatiza e toma as decisões sobre nosso
Programa}. Ela é formada pelo Coordenador do PPGE-USP e seu suplente,
por três docentes credenciados como orientadores do PPGE-USP e
vinculados ao Instituto de Biociências da USP, e uma representante
discente do PPGE- USP. Os membros da CCP têm um mandato de dois anos,
com exceção dos representantes discentes cujo mandato é de um ano, e são
eleitos por seus pares. \textbf{Assim, estudantes elegem seu/sua
representante, que tem o papel importantíssimo de estabelecer a
comunicação entre o corpo discente e a CCP}.

A CCP se reúne ordinariamente uma vez por mês, quando aprecia todas as
solicitações a ela encaminhadas, incluindo os trâmites para exames de
qualificação, defesa de dissertação ou de teses. Por isso, preste atenção
às datas das reuniões, para encaminhar estas e quaisquer outras
solicitações em tempo hábil. \textbf{As reuniões da CCP são abertas a
todas e todos estudantes e docentes do PPGE-USP}. As datas e horários
das reuniões estão em nossa página na internet, na seção \emph{Normas}
\textgreater{} \emph{Reuniões da Comissão}.

\subsection{Comissão Permanente para Ações Afirmativas - CoPAF}
A CoPAF é uma comissão auxiliar da CCP, que dá suporte à implementação
de ações afirmativas no PPGE-USP e zela para que estas ações sejam
efetivas assegurando permanência e pertencimento, através de:
\begin{itemize}
\item Estudar, planejar, propor e/ou implementar ações afirmativas; 
\item Acompanhar e avaliar as ações afirmativas, a fim de propor
aperfeiçoamentos e garantir que estas atendam às pessoas de direito;
\item Acompanhar e oferecer suporte a ingressantes por meio de ações
afirmativas, de maneira que tenham as condições e o apoio para se
desenvolver e se expressar plenamente;
\item Promover ações para fomentar uma cultura inclusiva no programa.
\end{itemize}

A CoPAF é por representantes discentes, docentes, de funcionários e de
coletivos de pessoas com direito às ações afirmativas. Para saber
mais, veja em nosso site a seção ``O Programa'' > ``Ações
Afirmativas''.

\chapter{Etapas de sua pós-graduação}

\section{Matrícula}

A sua matrícula inicial é feita na secretaria da Comissão de
Pós-Graduação do IBUSP (CPG-IB, veja instruções na página da
internet\footnote{\url{http://pos.ib.usp.br/}}). Pedidos oficiais,
como solicitações de exames de qualificação, defesas e prorrogações de
prazo devem ser protocolados na Seção de Protocolo do IB.

\textbf{Após sua matrícula inicial, toda sua vida acadêmica na
  pós-graduação é administrada por meio do Sistema Janus}
\footnote{\url{https://uspdigital.usp.br/janus/}}. É por meio do Sistema
Janus que você fará matrícula em disciplinas, acompanhará suas notas e
frequências, prazos, e resultados de solicitações. Assim que estiver
matriculada(o), crie uma conta no Janus e experimente navegar pelas
seções, para se familiarizar.

\textbf{Importante}: o Janus envia alguns recados importantes para o
seu e-mail USP. Por isso, mantenha seu e-mail USP ativo e verifique
periodicamente sua caixa de entrada.

Tanto mestrado como doutorado seguem ciclos semestrais. Cabe à aluna
ou aluno matricular-se a cada semestre que se inicia, usando o sistema
Janus. \textbf{Mesmo que você não pretenda cursar nenhuma disciplina em um
determinado semestre, deve fazer sua matrícula de acompanhamento} (ver
instruções no Sistema Janus).

\section{Disciplinas}
\label{sec:disciplinas}

Nossa filosofia é proporcionar aos estudantes liberdade para buscar as disciplinas
que julgar importantes para sua formação. Por isso, nosso Programa não
tem disciplinas obrigatórias, mas \textbf{há  disciplinas
  consideradas básicas para a formação de ecólogos que são oferecidas
  todos os anos}:


\begin{description}
%  \item[BIE5701] Prática em Pesquisa Ecológica\footnote{\url{http://ecologia.ib.usp.br/curso/}}, condensada, em julho.
  \item[BIE5793] Princípios de Planejamento e Análise de Dados em Ecologia\footnote{\url{http://labtrop.ib.usp.br/doku.php?id=cursos:planeco/}}, condensada, cinco semanas no primeiro semestre.
  \item[BIE5778] Ecologia de Comunidades\footnote{\url{http://ecologia.ib.usp.br/bie5778}}, condensada, quatro semanas no segundo semestre.
  \item[BIE5782] Uso da Linguagem R para Análise de Dados em Ecologia\footnote{\url{http://ecor.ib.usp.br/doku.php}}, condensada, três semanas no primeiro semestre.
  \item[BIE5786] Ecologia de Populações\footnote{\url{http://ecologia.ib.usp.br/bie5786}}, condensada, quatro semanas no primeiro semestre.
\end{description}

A disciplina \textbf{Preparação Pedagógica em Biologia (BIP5700)
  também é oferecida anualmente} e é \textbf{recomendável para alunos
  de doutorado bolsistas CAPES}. Se você usufruiu de bolsa de
doutorado CAPES em algum momento de seu doutorado, deverá realizar
estágio de docência que inclui cursar esta disciplina, ou disciplinas
equivalentes oferecidas por outros Programas de Pós-Graduação da USP
ou um dos vários ciclos de seminários que são oferecidos em outros
institutos da USP (ver item sobre PAE, pág. \pageref{subsec:pae}).

As demais disciplinas do PPGE-USP são oferecidas a cada dois ou três
anos. Graças à diversidade de áreas de pesquisas de nossos
professores, temos um leque amplo de disciplinas, incluindo
disciplinas oferecidas -- em caráter único -- por professores
convidados de outras instituições do Brasil e do exterior. Além disso,
você tem à disposição todas as disciplinas de pós-graduação oferecidas
na USP. Consulte nosso site para o catálogo das disciplinas do
PPGE-USP, e a lista das disciplinas oferecidas no semestre, no site da
CPG-IB (\url{https://pos.ib.usp.br}). Para as disciplinas de outros
Programas da USP, consulte o sistema Janus.  O PPGE também convalida
créditos de disciplinas de Programas de Pós-Graduação de outras
universidades, mediante solicitação da(o)  estudante.

Cabe a seu/sua orientador(a) e a você compor um plano de estudo com
as disciplinas que julgarem convenientes, que devem totalizar
\textbf{24 créditos para o mestrado e 15 créditos para alunos de
  doutorado} que cursaram mestrado. \textbf{Alunos de doutorado direto
  devem obter 39 créditos}.

Pelo menos dois terços dos créditos devem ser de disciplinas
oferecidas por cursos de pós-graduação de qualquer unidade da
USP. Créditos obtidos em disciplinas de outras universidades estão
sujeitos à aprovação da CCP. Adicionalmente, pode ser atribuído um 
crédito por sua participação no Programa de Aperfeiçoamento de Ensino
(PAE, pág. \pageref{subsec:pae}), a seu pedido.


\section{Comitês de Acompanhamento}
\label{sec:comites}

As alunas e os alunos devem ter um Comitê de Acompanhamento, com o
objetivo de auxiliar o planejamento, execução e defesa da dissertação
ou tese, bem como na formação geral da(o) aluna(o). O Comitê é formado
por orientador(a) e pelo menos outros dois pesquisadores ou
pesquisadoras com doutorado, que são escolhidos pela(o) própria(o)
aluna(o) e seu/sua orientador(a). O Comitê é uma instância consultiva
e de apoio à orientação e não de avaliação. O PPGE não obriga a
inclusão do nome dos integrantes do Comitê nas publicações que
derivarem da dissertação ou tese da(o) aluna(o), deixando este acordo
à critério de você, orientador(a) e seu comitê. As normas e
recomendações relativas aos Comitês de Acompanhamento estão em nossa
página na internet, seção \emph{Normas} \textgreater{} \emph{Do
  Programa} \textgreater{} \emph{Comitês}. Esteja atenta(o) aos
prazos:

\subsubsection{Prazos dos Comitês do Mestrado}
\begin{description}
\item[Indicação do Comitê:] até  5 meses após seu ingresso no Programa.
\item[Primeira reunião:] até o 6º mês após seu ingresso no Programa.
\item[Segunda reunião:] até  o 14º mês após seu ingresso no Programa.
\item[Terceira reunião:] até  o 24º mês após seu ingresso no Programa.
\end{description}

\subsubsection{Prazos dos Comitês do Doutorado}
\begin{description}
\item[Indicação do Comitê:] até 5 meses após seu ingresso no Programa.
\item[Primeira reunião:] até o 6º mês após seu ingresso no Programa.
\item[Segunda reunião:] até  o 18º mês após seu ingresso no Programa.
\item[Terceira reunião:] até  o 30º mês após seu ingresso no Programa.
\item[Quarta reunião:] no máximo 6 meses antes do prazo de depósito.
\end{description}

\subsubsection{Objetivos das reuniões de Comitê}
\begin{description}
\item[Primeira reunião:] Temas sugeridos para discussão com o comitê: (1) bases teóricas do projeto,
(2) delineamento amostral ou experimental do projeto, (3) cronograma do projeto, 
(4) sugestões de disciplinas a serem cursadas pelo(a) aluno(a).
\end{description}
\section{Exame de qualificação}
\label{sec:qualif}

O exame de qualificação avalia o progresso de seu conhecimento e
amadurecimento científico em sua área de pesquisa. A qualificação é
\textbf{obrigatória} apenas \textbf{para as alunas e alunos de doutorado} e
não há necessidade de obter créditos em disciplinas para sua
realização. O exame de qualificação consiste na entrega de:
\begin{enumerate}
\item Artigo científico relacionado ao projeto de Doutorado pronto para publicação no qual o aluno é o primeiro autor;
\item Carta de apresentação do artigo (\emph{cover letter}) ressaltando a importância da publicação; 
\item Planejamento da estrutura da tese e cronograma de trabalho até a defesa;
\item Histórico escolar no doutorado; e 
\item Fichas de avaliação semestrais com as atividades realizadas ao longo do doutorado.
\end{enumerate}

A avaliação será realizada por uma banca de três professores, que
arguirão a(o) aluna(o). O(A) orientador(a) não faz parte da banca
examinadora. Os exames de qualificação do doutorado e doutorado direto
diferem em relação aos prazos e regras. Quem é reprovado(a) no
exame de qualificação poderá repeti-lo apenas uma vez, devendo
realizar nova inscrição em até 90 dias contados a partir da data de
realização do primeiro exame. A reprovação no segundo exame leva ao
desligamento.


\section{Avaliação de desempenho}
\label{sec:conduta}

O seu desempenho acadêmico inclui, além dos resultados nas disciplinas
e no exame de qualificação, a condução adequada de seu projeto de
pesquisa, que resultará em sua tese ou dissertação. Há critérios
mínimos a atender, ou seu desempenho acadêmico e científico será
considerado insatisfatório, podendo resultar em seu desligamento do
curso. Além das situações previstas no Regimento de Pós-Graduação da
USP\footnote{\url{http://www.leginf.usp.br/?resolucao=resolucao-no-7493-de-27-de-marco-de-2018}},
as normas do PPGE consideram indicadores de desempenho insatisfatório:

\begin{itemize}
\item Descumprimento injustificado do projeto de pesquisa e respectivo
  cronograma de atividades;
\item Ausência injustificada às atividades do programa ou atividades
  de tese/dissertação por período superior a três meses;
\item Reprovação pelo(a) orientador(a), pela segunda vez (consecutiva
  ou não), do relatório de atividades, com justificativa sobre os
  aspectos relacionados à improdutividade da(o) aluna(o). A
  justificativa do(a) orientador(a) deverá ser submetida à CCP;
\item Acumular duas advertências, sem justificativa ou com
  justificativas não aceitas pela CCP, em função do descumprimento dos
  prazos do comitê de acompanhamento;
\item Procedimento irregular de natureza grave (e.g., plágio ou fraude, ver abaixo).
\end{itemize}

Constatada alguma dessas situações, a CCP irá avaliar e deliberar
sobre o desligamento, tendo a(o) aluna(o) o direito de apresentar suas
justificativas, no prazo de dez dias, caso julgue o desligamento
improcedente.

\subsection{Procedimentos irregulares}

O maior patrimônio do pesquisador são suas criações intelectuais, e
seu valor é reconhecido sob a premissa de que são criações
legítimas. O plágio é qualquer apropriação indevida de criações
intelectuais alheias e a fraude é forjar uma criação intelectual sem a
devida fundamentação.

Plágio e fraude são as duas falhas éticas mais graves que um
pesquisador ou pós-graduando pode cometer e que podem comprometer sua
carreira de forma irreversível. Por isso, \textbf{nosso Programa
  inclui plágio e fraude entre os procedimentos irregulares de
  natureza grave, que incorrem no desligamento, se comprovados}.

\subsubsection{Plágio}
O plágio em ciência ocorre, em geral, como apropriação de ideias,
dados ou textos de outros autores sem o devido crédito. Informe-se
para evitar este erro, pois cometê-lo inadvertidamente não atenua sua
responsabilidade. Há muito material a respeito que você pode consultar
para se informar melhor.

Como leitura inicial recomendamos a cartilha \emph{On Being a
  Scientist}\footnote{\url{http://www.nap.edu/catalog.php?record_id=4917}}
e o manual de boas práticas acadêmicas da
FAPESP\footnote{\url{http://www.fapesp.br/boaspraticas/}}. Além de
consultar esses materiais, é crucial que você converse com seu/sua
orientador(a) e comitê a respeito.

Podemos adiantar dois princípios simples para evitar o plágio em documentos científicos:

\begin{itemize}
\item Sempre que utilizar dados ou ideias de outras pessoas,
  creditá-los aos seus autores de maneira clara e inequívoca;
\item Caso utilize transcrições de textos de outros autores, no idioma
  original ou traduzidos, coloque o trecho transcrito entre aspas e
  indique o autor e obra de onde foram obtidos.
\end{itemize}


\subsubsection{Fraude}
\label{sec:fraude}

As fraudes científicas mais comuns são a alteração de resultados e a
criação de dados falsos. Uma premissa básica da nossa profissão é que
toda a atividade científica assenta-se na confiança mútua dos
pesquisadores de que toda a evidência empírica é legítima e não sofreu
uma intervenção para atender a expectativas ou interesses. O princípio
básico para evitar a fraude é simplesmente buscar coletar seus dados
de maneira isenta e nunca alterá-los para chegar a um resultado
desejado.  Além do material citado acima, recomendamos o artigo
\emph{¿Ecólogos o Ególogos? Cuando las ideas someten a los
  datos}\footnote{\url{www.scielo.org.ar/pdf/ecoaus/v19n2/v19n2a09.pdf}},
do ecólogo Alejandro Farji-Brener.

\section{Recesso laboral}
\label{sec:recesso-laboral}
O PPGE-USP reconhece a necessidade dos estudantes terem um recesso
laboral a cada ano, que deve ser planejado com antecedência com o
orientador ou a orientadora, e apresentado em seu
cronograma de trabalho. Veja as recomendações da CCP em nosso site
(seção \emph{O programa} \textgreater{} \emph{Comissão Coordenadora}
\textgreater{} \emph{Deliberações}).

\section{Conclusão de sua pós-graduação}
\label{sec:conclusao}

Tendo cumprido todas as atividades obrigatórias (número mínimo de
créditos, qualificação e reuniões do comitê), você pode concluir seu
curso. Isto é feito com o depósito de sua dissertação/tese na
Secretaria de Pós-Graduação do Instituto de Biociências dentro do
prazo máximo de integralização

\subsection{Prazos}

O prazo de integralização é contado a partir de seu ingresso
(primeira matrícula):

\begin{description}
\item[Mestrado:] até 28 meses após o ingresso.
\item[Doutorado:] até 48 meses após o ingresso.
\item[Doutorado direto:] até 60 meses após o ingresso.
\end{description}

\subsection{Depósito da dissertação ou tese}
\label{sec:depos-da-diss}

O depósito de tese é feito pelo sistema Janus, na seção \emph{Depósito
  digital}. Além do arquivo em pdf da tese, é preciso anexar
formulário de
depósito\footnote{\url{http://pos.ib.usp.br/arquivos/formularios.html}},
devidamente assinado por estudante e orientador(a), e também a
sugestão de banca examinadora. No dia anterior ao depósito envie, para
o e-mail \href{mailto:cpg@ib.usp.br}{\nolinkurl{cpg@ib.usp.br}}, o
resumo e o \emph{abstract} da dissertação/tese (em arquivo rtf ou
doc). Os formulários e instruções detalhadas para depósito estão no
site da CPG-IB, na seção \emph{Formulários}.

O depósito comprova a conclusão de sua tese ou dissertação dentro do
prazo de integralização. Em seguida, a CCP indicará a comissão
julgadora e a data de sua defesa será marcada. Para informações
detalhadas sobre prazos, composição da banca, tempo de apresentação e
critérios de avaliação das dissertações/teses, consulte as páginas do
PPGE-USP e da CPG-IB.

\subsection{Prorrogação de prazo}

\textbf{O prazo para depósito da dissertação/tese pode ser prorrogado
  em casos excepcionais, por período não superior a 60 dias}. Para
solicitar prorrogação, o aluno deverá enviar à CCP um requerimento
assinado e com parecer circunstanciado do(a) orientador(a), com
ciência do coordenador do PPGE-USP, acompanhado de justificativa da
solicitação, versão preliminar da dissertação ou tese e cronograma das
atividades a serem desenvolvidas no período de prorrogação.

De acordo com os critérios vigentes, a CCP entende como
excepcionalidade as seguintes situações:

\begin{itemize}
\item Problemas de saúde que tenham prejudicado o desenvolvimento da
  dissertação ou tese, comprometendo o cronograma original;
\item Problemas pessoais graves, como falecimento ou doença grave de
  familiares próximos;
\item Imprevistos no trabalho de campo ou laboratório que atrasaram o
  cronograma original.
\end{itemize}

As seguintes situações não caracterizam excepcionalidade e, portanto,
não são consideradas pela CCP em pedido de prorrogação de prazo de
depósito:

\begin{itemize}
\item Coleta de informações adicionais não previstas no cronograma
  original;
\item Análises adicionais não previstas no cronograma original;
\item Atrasos no cronograma em decorrência de outras atividades,
  remuneradas ou não.
\end{itemize}

\chapter{Bolsas e auxílios}

\section{Bolsas de estudo}
\label{sec:bolsas}

Há duas maneiras de se obter uma bolsa de estudos em nosso Programa:
bolsas da cota institucional, que são geridas pelo PPGE-USP, e bolsas
atribuídas diretamente a seu/sua orientador(a).

\subsection{Bolsas institucionais}

O Programa recebe do CNPq e da CAPES bolsas institucionais, mas não há
garantia de que elas sejam suficientes.
Caso não haja bolsas para todos, os alunos melhor classificados no
exame de ingresso têm precedência na concessão das bolsas
institucionais. A classificação é feita pelo resultado do exame de
ingresso, e está publicada na página do PPGE-USP, assim como o
cronograma de disponibilidade de bolsas. \textbf{Assim que uma bolsa
  da cota do PPGE-USP torna-se disponível, os alunos da lista de
  espera são consultados sobre seu interesse na bolsa, estejam
  matriculados ou não.}  Os alunos não matriculados devem
matricular-se para que possam usufruir da bolsa.

\textbf{Se você foi selecionado para uma bolsa institucional, deve
  conhecer e aceitar as normas de concessão estabelecidas pelas
  agências financiadoras} (CNPq e CAPES) por meio da assinatura do
Termo de Compromisso. Essas normas incluem dedicação exclusiva à
pós-graduação e ausência ou suspensão de vínculo empregatício. Pode-se
solicitar permissão do Programa para trabalhar em outra atividade
relacionada à tese, por até doze horas semanais. Algumas agências,
como é o caso da CAPES, de fato exigem o envolvimento do bolsista em
atividades didáticas. \textbf{Portanto, conheça as normas de concessão
  antes da assinatura do Termo de Compromisso e informe imediatamente
  caso você não possa mais atendê-las} (por exemplo, em caso de
estabelecer vínculo empregatício ou ter recebido outra bolsa).

Além das normas de cada agência, o PPGE-USP tem a norma interna de que
\textbf{a bolsa institucional de mestrado só pode ser usufruída até o
  24° mês, a contar da matrícula. Para o doutorado, a bolsa só pode
  ser usufruída até o 48° mês de curso.} Vencidos esses prazos, a
bolsa é transferida para a próxima pessoa da lista de espera, mesmo
que o(a) antigo(a) bolsista ainda não tenha concluído sua
pós-graduação. Com isso, busca-se estimular a conclusão da
pós-graduação dentro dos prazos e permitir que mais estudantes tenham
acesso às bolsas.

As normas para nossa cota institucional, bem como a disponibilidade de
bolsas e a lista de espera estão no site do PPGE-USP, na seção
\emph{Recursos} \textgreater{} \emph{Bolsas}.

\subsection{Bolsas individuais}

Além de bolsas institucionais, há bolsas que podem ser pleiteadas
pelo(a) orientador(a) diretamente junto a outras instituições, como a
FAPESP. Converse com seu/sua orientador(a) sobre a possibilidade de
solicitar essas bolsas. O PPGE-USP considera extremamente positivo que
seus/suas alunos(as) e orientadores(as) pleiteiem diretamente bolsas
em órgãos de fomento, pois isto alivia a demanda pelas bolsas
institucionais e é um indicador da qualidade de seus
quadros.


% \subsection{Bolsas emergenciais}

% Caso você não tenha sido contemplado com uma das bolsas institucionais e esteja aguardando algum pedido de bolsa individual, há
% ainda a bolsa emergencial, que é uma cota gerenciada pela Pró-Reitoria de Pós-
% Graduação da USP. \textbf{Podem solicitar a bolsa emergencial alunos que fizeram pedidos
% de bolsas individuais a órgãos de fomento.} Para solicitar uma bolsa emergencial, seu
% orientador deve enviar uma solicitação à CCP, informando a qual órgão de
% fomento já foi solicitada uma bolsa. Deve-se anexar o protocolo de solicitação da bolsa,
% e alguns outros documentos. A Secretaria da PPGE fornece as orientações
% completas aos interessados. Toda a documentação dos interessados em bolsas
% emergenciais é então encaminhada pela CCP para a Pró-Reitoria de Pós-Graduação da
% USP. Não há nenhuma garantia que as solicitações sejam aceitas, pois o número de
% bolsas emergenciais varia de ano para ano e a cota total de bolsas emergenciais
% atende a todos os programas de pós-graduação da USP, o que torna o processo de
% obtenção bastante competitivo.

\section{Auxílio financeiro}

\subsection{Verba PROEX}

O PPGE-USP recebe da CAPES todos os anos verba do Programa de
Excelência Acadêmica (PROEX
\footnote{\url{https://www.gov.br/capes/pt-br/acesso-a-informacao/acoes-e-programas/bolsas/bolsas-no-pais/proex}}),
para apoio de suas atividades. \textbf{O plano de aplicação do PROEX é
  aprovado pela CCP a cada ano e está publicado em nossa página na
  internet} (seção \emph{Recursos} \textgreater{} \emph{Verba PROEX} \textgreater{}
\emph{Informações Gerais}), bem como um balancete das aplicações.

Do total de verba PROEX disponível, a CCP reserva parte para pagamento
de %passagens e diárias para participantes de bancas de defesa e despesas gerais e
gastos administrativos do Programa. O restante, que é a maior parte, é
destinado ao auxílio às atividades diretamente relacionadas ao
trabalho de tese ou dissertação de cada aluna(o), como relacionamos a
seguir.

\subsection{Auxílio à pesquisa de alunas e alunos
}\label{auxilio-pesquisa-de-alunas-e-alunos}

Você pode solicitar recurso PROEX para atividades de sua pós-graduação como:

\begin{itemize}
\item Aquisição de material de consumo para sua pesquisa;
\item Despesas com trabalho de campo e coleta de dados;
\item Visita a coleções e especialistas;
\item Participação em congressos, cursos ou estágios fora da
  instituição (seja no Brasil ou no exterior).
\end{itemize}

A lista completa de itens financiáveis pelo PROEX está na Portaria
publicada pela CAPES. Em nosso Programa, \textbf{uma Comissão formada
  por alunas e alunos abre os editais e avalia os pedidos enviados,
  seguindo regras de avaliação estipuladas pelos próprios
  estudantes}. Há prazos para solicitar o recurso, por meio de um
formulário online \footnote{\url{http://siad-ecologia.ib.usp.br/}}.
Portanto, preste atenção aos editais que forem abertos. Além disso,
para facilitar o entendimento do documento pelo qual se realizam
pedidos de financiamento (via reembolso), a Comissão PROEX preparou um
resumo das regras dando ênfase às dúvidas mais frequentes. Para saber
mais, veja o vídeo de apresentação da Comissão Discente PROEX em nosso
canal do YouTube \footnote{\url{https://youtu.be/VlWq-LnhNHg}}, e
entre em contato com a Comissão PROEX através do endereço de email
\href{mailto:proex.eco@gmail.com}{\nolinkurl{proex.eco@gmail.com}}.

\subsection{Auxílio à publicação}

O PPGE-USP reserva parte da verba PROEX para pagamento de taxa de
publicação e revisão de inglês de manuscritos de seus alunos,
professores e egressos recentes. A verba é distribuída por
meio de editais periódicos. Consulte a página do programa, seção
\emph{Recursos} \textgreater{} \emph{Verba PROEX} \textgreater{} \emph{Auxílio a
Publicações}.


% \subsection{Verbas para participação em congressos}

% A Pró-Reitoria de Pós-Graduação da USP oferece auxílio para participação em
% eventos, tais como congressos, simpósios e outras atividades científicas e
% acadêmicas. Para auxílios em eventos no exterior, são aceitos somente pedidos de
% alunos de doutorado. Para eventos nacionais são aceitos alunos de mestrado e
% doutorado. Toda documentação encaminhada à CPG deverá ser entregue na Seção de  Protocolo do IB
% 80 dias antes da realização do evento. Para que seu pedido possa constar na ordem do
% dia das reuniões mensais da CPG, é importante que a documentação seja entregue no
% protocolo no IB-USP até a 4ª feira que antecede a reunião. Para mais informações,
% consulte a página da Pró-Reitoria de Pós-Graduação\footnote{\url{http://www.usp.br/prpg/pt/interna1/participEventos.html}}.

\chapter{Oportunidades}
\label{sec:oport}

\section{Estágios no exterior}
Há muitos programas de apoio que fornecem bolsas, passagens e auxílios
de instalação. Visite as páginas da Agência de Cooperação Acadêmica
Nacional e Internacional da USP para conhecer
\footnote{\url{http://www.usp.br/internationaloffice/}} e da
Pró-reitoria de Pós-Graduação da USP
\footnote{\url{http://www.prpg.usp.br}}. Abaixo seguem os principais
programas (fique atento(a), pois novas oportunidades surgem
constantemente).
\begin{description}
\item Programa de Internacionalização (PRInt), financiado pela
  CAPES\footnote{\url{https://sites.usp.br/print}}, que inclui bolsas
  para estágios no exterior e para visitas de curta duração.
\item Bolsas Sanduíche no Exterior, financiadas pelo
  CNPq\footnote{\url{http://cnpq.br/bolsas-no-exterior1}}.
\item Bolsa Estágio de Pesquisa no Exterior (BEPE), possível para
  bolsistas FAPESP\footnote{\url{http://www.fapesp.br/bolsas/bepe/}}.
\end{description}

\section{Programa de Aperfeiçoamento de Ensino}
\label{subsec:pae}

A USP mantém um programa de estágio supervisionado em docência, o
PAE. Participando do PAE você auxiliará em disciplinas de graduação,
sob a supervisão do professor responsável.  Antes disso, você deve
fazer a preparação pedagógica. Os participantes do PAE têm direito a
um crédito pela atividade e podem receber auxílio financeiro, sem
prejuízo da bolsa de pós-graduação. É uma ótima oportunidade de
treinamento em docência na graduação. \textbf{O PAE é obrigatório para
  alunos de doutorado com bolsa CAPES}.

As inscrições são semestrais e ocorrem em outubro, para realizar o
estágio no 1° semestre do ano e em maio, para realizar estágio no 2°
semestre. Mais informações na página do PAE, no site da Pró-Reitoria
de Pós-Graduação \footnote{\url{http://www.prpg.usp.br}}.

\section{Seminários EcoEncontros}
\label{subsec:seminarios}
O EcoEncontros (seminários do PPGE-USP) é uma iniciativa do corpo
discente, em vigor desde 2008. \textbf{As palestras são realizadas
  semanalmente. }

Todos os alunos podem se inscrever para ministrar uma palestra, que
pode versar sobre resultados de um trabalho já finalizado ou sobre o
projeto de mestrado ou doutorado em desenvolvimento. Os seminários são
uma ótima oportunidade de intercâmbio de ideias e um excelente canal
de comunicação entre estudantes e docentes do Programa.

Para conhecer mais visite \url{http://ecoencontros.ib.usp.br}.  Esteja
também atento(a) às mídias sociais do EcoEncontros e aos cartazes nos
murais do prédio no qual o Departamento de Ecologia se localiza e, se
desejar, entre em contato pelo correio eletrônico
\href{mailto:ecoencontros@ib.usp.br}{\nolinkurl{ecoencontros@ib.usp.br}}.

\section{Café Existencial}
\label{cafe-existencial}

O \emph{Café Existencial} é uma iniciativa do corpo discente de nosso
Programa, cujas atividades foram oficialmente iniciadas em 2017. Em
reuniões quinzenais, as alunas e os alunos se encontram para discutir
questões importantes para o desenvolvimento de seu trabalho, tais
como: saúde mental na pós-graduação, procrastinação e organização do
tempo, relação aluno-orientador, balanço entre trabalho e vida
pessoal, perspectivas futuras de atuação, entre outros temas. Assim, o
\emph{Café Existencial} se propõe a abrir um espaço em que sejam
discutidos quaisquer assuntos que possam atender a demandas de
estudantes que não aquelas diretamente ligadas a conteúdo sobre
Ecologia ou método científico, indo além dos espaços de discussão
tradicionalmente existentes em ambientes acadêmicos mundo afora.

O \emph{Café Existencial} é uma construção coletiva que demanda a
participação do maior número possível de alunas e alunos. As reuniões
não têm forma de discussão pré-estabelecida e são mediadas pelos
próprios estudantes. Para saber mais, veja o vídeo de apresentação do
Café Existencial em nosso canal do YouTube
\footnote{\url{https://youtu.be/TqeAue4V4l4}}. Para integrar a lista
de e-mails do Café Existencial e receber informações sobre novas
reuniões solicite cadastro através do seguinte link:
\href{mailto:cafeexistencial-ecousp+subscribe@googlegroups.com}{\nolinkurl{cafeexistencial-ecousp+subscribe@googlegroups.com}}.

\section{EcoEscola}
\label{ecoescola}

A EcoEscola é um curso de extensão anual gratuito realizado na
Universidade de São Paulo e organizado pelo corpo discente e
pesquisadores de Pós-doutorado do Programa, sob a supervisão da
Profª. Daniela Scarpa. O curso busca:

\begin{itemize}
\item Fornecer ferramentas teóricas e práticas em concepção e execução
  de projetos de pesquisa em ecologia para graduandos, recém-graduados
  e professores de ensino fundamental e médio;
\item Divulgar as áreas de pesquisa em ecologia que estão em andamento
  na USP;
\item Dar uma formação complementar em magistrado, elaboração e
  orientação para pós-graduandos do Programa que participem da
  organização;
\item Fomentar a troca de ideias entre alunas e alunos de graduação e
  de pós-graduação interessados em pesquisa e ensino em ecologia.
\end{itemize}

Para conhecer mais e participar veja o vídeo de apresentação da
EcoEscola em nosso canal do YouTube
\footnote{\url{https://youtu.be/OEGGA8sSF-c}}, e visite o site 
\url{https://ecoescola.ib.usp.br/}.

\section{Curso preparatório para ingressantes}
É curso preparatório para o exame de ingresso no mestrado e doutorado
do Programa de Pós-graduação em Ecologia, destinado para pessoas
contempladas pelas políticas de ações afirmativas do programa. É mais
uma iniciativa dos estudantes do PPGE-USP, para qual todos os estudantes
podem contribuir.

Mais informações no site do curso: \url{https://ecocurso.wixsite.com/ecocurso}

\chapter[Boas práticas acadêmicas]{Boas práticas acadêmicas}
\label{sec:boasprat}

A pós-graduação é a etapa de nossos estudos em que fica mais evidente
que a(o) estudante é a(o) protagonista de seu aprendizado. Todos os
recursos e ajuda fornecidos por orientadoras(es), professoras(es) e
equipe do Programa só serão efetivos se você tomar para si a
responsabilidade por sua formação. Há, portanto, mudanças importantes
de postura que devem ocorrer na pós-graduação, relacionadas a
definir-se como um bom profissional. A seguir, discutimos as
principais, muitas das quais podem já estar em curso para você. Vale
ressaltar, entretanto, que todas as mudanças são aspectos que demandam
constante reflexão e aperfeiçoamento, e prosseguem por toda a nossa
vida profissional.

\section[Comportamentos adequados no ambiente de trabalho]{Comportamentos adequados no ambiente de trabalho \footnote{Orientações preparadas pela Comissão IB Mulheres, adaptado para contemplar outros grupos que são alvo frequente de discriminação e assédio.}} 

Existem vários tipos de comportamentos que são inapropriados e que
podem afetar o bem-estar na sala de aula, nos laboratórios e nas
relações de trabalho. Essas atitudes são danosas, trazem sentimentos
de vulnerabilidade e isolamento profissional, frequentemente atingindo
mais as mulheres, pessoas de outros grupos que sofrem discriminação,
e/ou em posição de subordinação em uma hierarquia.

Uma vez que vários desses comportamentos ainda são comuns no nosso
dia-a-dia, apresentamos uma lista que pode ajudar a identificá-los,
evitá-los e denunciá-los.

\begin{description}

\item[comportamento intimidador:] palavras abusivas, intimidadoras,
  humilhantes ou ameaçadoras podem gerar estresse e
  constrangimento. Caracterizam assédio moral e as vítimas são
  comumente grupos vulneráveis e minorias sociais.

\item[piadas de mau gosto:] qual o intuito da ``brincadeira''? Se ela
  faz alguém se sentir mal, a piada está deixando de cumprir seu
  intuito. A falta de reação pública não significa aceitação, pois nem
  sempre há espaço para a pessoa constrangida se manifestar. No fundo,
  essas ``brincadeiras'' só reforçam estereótipos e vieses, explícitos ou
  implícitos.

\item[toques inconvenientes:] contatos físicos invasivos, abusivos e
  insistentes geram desconforto e constrangimento, devendo ser
  evitados, sendo comum viés de gênero e de posição hierárquicas na
  frequência e tipo desse contato.

\item[imagens ofensivas ou que reforçam estereótipos e preconceitos:]
  deve-se tomar cuidado com as imagens veiculadas em apresentações e
  palestras, evitando aquelas que reforçam estereótipos de gênero
  (como a objetificação da mulher e homofobia), de raça ou qualquer
  outro.

\item[perseguição (\emph{“stalking”}):] importunação, perseguição obsessiva e
  insistente devem ser denunciados, não sendo aceitáveis em nossa
  comunidade.

\item[tratamento discriminatório:] deve-se dar chances e oportunidades
  iguais a todos(as) em ambientes de trabalho, evitando associações
  entre tarefas específicas e estereótipos de qualquer tipo, os quais
  têm prejudicado a formação profissional de pessoas em grupos que
  sofrem discriminação, como as mulheres, negros, indígenas, LBGTQI+.
\end{description}

Se você vivenciou ou presenciou situações como as descritas acima,
busque apoio institucional.  A Secretaria do PPGE trata com sigilo,
seriedade e respeito qualquer caso desta natureza, relatados pelos(as)
afetados(as) ou terceiros(as). Os(As) representantes discentes junto ao
PPGE também têm experiência em ajudar nestes casos. Nosso Instituto
tem ainda a comissão IB Mulheres (mulheres@ib.usp.br), para casos de
discriminação e violência de gênero
(\url{https://mulheres.ib.usp.br/}).

\section[A pesquisa não tem uma jornada de trabalho usual]{A pesquisa não tem uma jornada de trabalho usual \footnote{Esta seção e as seguintes deste capítulo são livremente inspiradas na resenha "On being a successful graduate student in the sciences", do Dr. John N. Thompson (University of California, Santa Cruz, E-mail: \url{thompson@biology.ucsc.edu}. O John é um colaborador regular de nosso programa e somos gratos a ele pela permissão de uso de seu material.}}

Não trate sua pós-graduação como um trabalho de horário fixo de 40
horas semanais, pois a pesquisa é um trabalho por demanda, não por
jornada. Todo pesquisador e pesquisadora têm liberdade para gerir seu
horário, pois há certas tarefas que exigem períodos de jornadas longas e
não usuais, como a coleta de dados, uma disciplina intensiva, ou
finalizar o texto da qualificação. Há um risco grande de confundir
esta liberdade de gestão do horário e a sucessão de rotinas diferentes
com falta de rotina. Por isso, capacidade de planejamento é tão vital
para cientistas como suas ideias, seu conhecimento ou sua
capacidade de trabalho.

Outro equívoco comum é a crença que trabalho duro, como um campo
especialmente difícil, é sinônimo de qualidade de uma pesquisa. Sem
dúvida, há momentos de trabalho intenso na pós-graduação, mas eles só
farão sentido dentro de um planejamento maior. Para o exame de
ingresso você apresentou um projeto com um cronograma de
execução. Reavalie este cronograma cuidadosamente e verifique se todas
as metas para concluir sua pós-graduação estão definidas com a
clareza, se a cada meta está atribuído um prazo realista e estabeleça
critérios concretos para avaliar se você alcançou cada uma de suas
metas. Além de dar sentido aos momentos críticos de sua pós-graduação,
com um bom cronograma você pode se preparar para eles e ainda ter
tempo para seu lazer, descanso e vida pessoal.

\section{Ordem e organização}

É evidente que não se pode chegar a fazer tudo o que se deseja ao
longo da vida, simplesmente porque não há tempo suficiente. O
problema, portanto, é o que fazer e o que deixar de fazer. Essa
decisão não deve ser tomada por capricho. Trata-se de uma tentação
comum fazer o urgente antes do importante, o fácil antes do difícil, o
que termina rápido antes do que requer um esforço de longa duração. A
ordem e a organização, que são virtudes que também dependem muito dos
orientadores, são a melhor forma de lidar com esses conflitos. Alguns
detalhes informativos das virtudes da ordem e da organização são, por
exemplo, a pontualidade e o cumprimento de prazos. Quando não há ordem
e organização na cabeça, acabamos sempre por escolher as atividades
que mais nos apetecem ou aquilo que nos parece urgentíssimo, mas que
pode não ser a melhor escolha no momento. Como tudo mais em nossa
formação, trata-se de um aprendizado contínuo. Busque conversar a
respeito com colegas, docentes, orientador(a) e seu comitê.

\section{Relação aluno(a)/orientador(a)}

A relação com o orientador ou a orientadora tem um impacto muito
grande na formação e no desenvolvimento do trabalho de pós-graduandos,
mas ela pode se tornar disfuncional quando as expectativas em relação
ao papel de cada um não estão claras ou são divergentes. Pensando em
minimizar eventuais problemas, listamos algumas dicas que podem guiar
essa relação \footnote{As dicas aqui apresentadas são baseadas neste
  texto de Adrian Sgarbi, do Pesquisatec Blog: {\url{goo.gl/PFVRGW}}}.

\subsection{O que esperar do(a) orientador(a)?}

	De modo geral, é papel do(a) orientador(a):
	
\begin{enumerate}
\item Auxiliar na delimitação do tema de pesquisa.
  % \footnote{embora
  %   relevantes tanto para mestrado quanto para doutorado, esses
  %   tópicos são especialmente importantes durante mestrados, pois
  %   espera-se dos estudantes de doutorado maior autonomia no
  %   desenvolvimento da pesquisa.\label{fn:om1}}
\item Indicar e conversar sobre referências bibliográficas relevantes,
  principalmente se a(o) aluna(o) está iniciando a pesquisa em uma
  área nova para ela(e) %\footref{fn:om1}.
\item Assumir co-responsabilidade pelo desenvolvimento do
  trabalho %\footref{fn:om1}.
\item Opinar sobre formas de melhorar o trabalho em
  andamento %\footref{fn:om1}.
\item Auxiliar na realização de pedidos de financiamento para o
  projeto %\footref{fn:om1}.
\item Explicar o que se espera de situações importantes do programa,
  como as reuniões de comitê e a qualificação\footref{fn:om1}.
\item Auxiliar na preparação para qualificação e defesa.
\item Participar de reuniões para tratar do desenvolvimento do
  trabalho (a frequência das reuniões deve ser acordada previamente
  entre aluno(a) e orientador(a))\footref{fn:om1}.
\item Ser pontual nos encontros marcados e evitar interrupções externas.
\item Ter em mente que a(o) aluna(o) não é obrigada(o) a trabalhar aos
  finais de semana e feriados.
\item Sugerir disciplinas potencialmente relevantes para a(o) aluna(o). 
\item Revisar diferentes versões do manuscrito ao longo do
  desenvolvimento do trabalho conforme for necessário (mesmo que as
  versões contenham apenas algumas seções).
\item Ler a versão final do manuscrito.
\end{enumerate}


\subsection{O que não esperar do(a) orientador(a)?}
Você não deve esperar que a(o) orientador(a):
\begin{enumerate}
\item Seja sua/seu amiga(o), já que essa é uma relação profissional
  (embora nada impeça laços de amizade).
\item Responda a todas as suas dúvidas.
\item Resolva problemas que você tenha com outras pessoas do programa
  (embora possa conversar sobre isso com ele ou ela).
\item Lembre os prazos que você deve cumprir.
\item Procure você para marcar reuniões.
\item Trabalhe aos finais de semana ou durante as férias.
\item Fique satisfeito quando você não segue as decisões sobre sua
  pesquisa ou pós-graduação que foram acordadas entre vocês.
\item Fique satisfeito com atrasos para as reuniões.
\item Fique satisfeito ao receber textos produzidos sem o devido
  cuidado de reflexão e redação.
\end{enumerate}

\subsection{O que você deve fazer como estudante}
\begin{enumerate}
\item Participar ativamente da delimitação do tema e da forma de
  desenvolvimento de sua pesquisa.
\item Buscar, ler e discutir com seu/sua orientador(a) as referências
  bibliográficas relevantes para o desenvolvimento de seu projeto.
\item Conhecer e respeitar as normas do Programa e da Universidade de São Paulo.
\item Participar de reuniões para tratar do desenvolvimento de seu
  trabalho (a frequência das reuniões deve ser acordada previamente
  com seu/sua orientador(a) ).
\item Ser pontual nos encontros marcados e evitar interrupções externas.
\item Participar ativamente da definição e nas reuniões do seu comitê
  de acompanhamento.
\item Trabalhar nos momentos em que se espera que você esteja
  desenvolvendo seu projeto, respeitando-se horários e períodos de
  descanso e lazer.
\item Informar-se previamente à matrícula sobre disciplinas que sejam
  de seu interesse (para sua formação e para o desenvolvimento de seu
  projeto) e discutir as melhores opções com seu orientador.
\item Lembrar prazos importantes do Programa.
\item Entregar com antecedência material que será discutido nas
  reuniões de comitê de acompanhamento ou em reuniões exclusivas com
  sua/seu orientador(a).
\item Entregar versão final de seu projeto com antecedência, de modo
  que sua/seu orientador(a) possa revisá-la adequadamente.
\item Contribuir para o seu desenvolvimento pessoal e principalmente
  do Programa participando das diferentes atividades coletivas
  fomentadas pelo programa e/ou pelo corpo discente, como EcoEscola,
  EcoEncontros, Café Existencial, Comissão PROEX, reuniões convocadas
  pela representação discente, eventos informais com outros docentes e
  estudantes (como "cafezinhos"), etc. Apesar da participação ser
  voluntária, todas essas instâncias são construídas coletivamente e
  solicitações de mudanças e melhorias no programa são tão fortes e
  legítimas quanto maior for a participação nos diferentes grupos
  e espaços democráticos disponíveis.
\end{enumerate}

Sugerimos que, antes de se iniciar uma pós-graduação, orientador(a) e
estudante conversem sobre as expectativas de cada um(a), usando como
referência os tópicos acima. Por exemplo, é interessante que desde o
início esteja acordado entre ambos a responsabilidade de cada parte
pelo apoio financeiro e material que a pesquisa receberá. Quanto mais
claro estiver para ambos(as) as responsabilidades de cada um na
formação do estudante e na realização do projeto de pesquisa, maiores
as chances de a(o) aluna(o) tirar melhor proveito do curso de
pós-graduação e alcançar uma formação plena e satisfatória.

Também é importante lembrar que os tópicos acima não são exaustivos,
mas já mostram o quão complexa é a relação entre aluno(a) e
orientador(a). Esta relação depende muito de aspectos intrínsecos às
personalidades dos envolvidos, ao ambiente de trabalho, aos objetivos,
expectativas e valores de cada uma das partes e, eventualmente, à área
de pesquisa em que o laboratório se insere (tendo em vista que
diferentes áreas de pesquisa têm diferentes culturas de
orientação). A relação em contextos de orientação acadêmica é um
debate que tem ganhado espaço nas últimas décadas, inclusive como tema
de pesquisa em educação. Se você deseja se aprofundar no assunto,
sugerimos três trabalhos que abordam esta temática:
\begin{enumerate}
\item Bastalich (2015). Content and context in knowledge production: a
  critical review of doctoral supervision literature. \textit{Studies
    in Higher Education}, DOI:
  {\url{http://dx.doi.org/10.1080/03075079.2015.1079702}}
\item Becker \textit{et al}. (2010). Approaches to doctoral
  supervision in relation to student
  expectations. {\url{http://www.brand.lth.se/fileadmin/brandteknik/utbild/Pedagogik/resurser/Report__final_version_.pdf}}
\item Acker \textit{et al}. (1994). Thesis supervision in the social
  sciences: managed or negotiated. \textit{Higher Education}, 28:
  483-498.
\end{enumerate}

\section{Você faz parte de um laboratório}

Durante a sua pós-graduação você fará parte de um laboratório que
possivelmente incluirá, além de seu/sua orientador(a), pesquisadores
de pós-doutorado e alunos(as) em diferentes fases de suas carreiras,
desde estudantes de iniciação científica até estudantes no fim do
doutorado. Além disso, alguns de nossos docentes possuem laboratórios
coletivos, dos quais participam estudantes e professores com
interesses distintos e, muitas vezes, complementares ao seu. Aproveite
essa oportunidade para conhecer outros pesquisadores e aprender mais
sobre a pesquisa que eles desenvolvem. Estude cuidadosamente os
trabalhos publicados pelo laboratório de pesquisa do qual você faz
parte. Afinal, você escolheu trabalhar neste laboratório porque as
questões que este grupo de pesquisadores quer responder são próximas
às do seu próprio interesse, não é mesmo?

Complementarmente, informe-se sobre os trabalhos e atividades
desenvolvidos em outros laboratórios. Integrar-se às atividades de
outros laboratórios pode ser uma excelente forma de aprimorar sua
formação e, ao mesmo tempo, contribuir com o desenvolvimento de outros
laboratórios.

\section{Teoria é fundamental}

Segundo o economista Edward E. Lawler \emph{"teoria sem dados é
  fantasia, mas dados sem teoria é caos"}. Durante a pós-graduação, é
importante que você identifique as disciplinas que irão ajudá-lo a
entrar em contato com a literatura pertinente da sua área de estudo e
a lê-la de maneira crítica. Além disso, desenvolva o hábito de
consultar os principais periódicos das áreas relacionadas ao seu
projeto. A maior parte dos principais periódicos sobre ecologia possui
um sistema de notificação por correio eletrônico que permitirá você
sempre se manter informado sobre o que está acontecendo na sua
área. Note, entretanto, que para desenvolver seu projeto, você
provavelmente necessitará não só de muito conhecimento sobre a área
principal do projeto, mas também de outras áreas relacionadas.

\section{Suas perguntas são interessantes ou triviais?}

Formular boas perguntas é um desafio em qualquer área da ciência,
especialmente no início de carreira. Diante dessa
dificuldade, é importante que jovens ecólogos tenham discernimento
para responder uma questão crucial: estou respondendo perguntas
interessantes ou minhas perguntas são triviais? A resposta depende do
momento de sua formação, e de vários outros contextos, como o estado
da área de conhecimento de sua pesquisa.

Chegar a uma pergunta relevante depende da ajuda de colegas mais
experientes, a começar pelo(a) seu(sua) orientador(a), e seu Comitê de
Orientação. O fundamental é ter clareza de todos os elementos que
contam para a relevância de uma questão de pesquisa para a sua
formação e para sua área de conhecimento.  Depende também da
viabilidade de se investigar esta questão, dentro dos prazos e
recursos que você terá.  Um excelente roteiro geral para essa reflexão
é o artigo:

Alon, U., 2009. How to choose a good scientific problem. \emph{Molecular cell, 35(6)}, pp.726-728.


Especificamente no contexto da pesquisa em ecologia e evolução, há
dois tipos de perguntas que são em geral consideradas triviais, e que
você deveria evitar na sua dissertação ou na sua tese:

\begin{enumerate}
\item Há infinitas possibilidades de comparação de duas ou mais
  situações, momentos, ou locais. Por isso, estudos comparativos
  precisam de uma justificativa bem fundamentada da sua relevância e
  contribuição potencial para a área de conhecimento. Em geral, essa
  justificativa permite enunciar hipóteses respeito do que se espera
  encontrar e por qual razão. Sem isso, sua questão resume-se a
  ``comparar por comparar'', o que é uma trivialidade.
\item A ausência de informações biológicas sobre um determinado
  organismo não é razão suficiente para justificar um estudo. Existem
  mais de 15 milhões de espécies no planeta e temos informação
  biológica para uma parcela ínfima deste total. Portanto, a falta de
  conhecimento sobre uma espécie é a regra e não é um bom argumento. É
  crucial que você tenha claro outras razões que fizeram você escolher
  uma espécie ou táxon em particular para investigar, entre os
  milhares pouco estudados. O mesmo argumento se aplica a escolha de
  ambientes ou sistemas de estudo.
\end{enumerate}

\section{A resposta para a pergunta “Com o que você trabalha?”}

Muitas vezes perguntamos aos nossos colegas com o que eles trabalham e
recorrentemente ouvimos como resposta o nome de uma espécie ou de um
táxon, um tipo de ambiente, um local, alguma interação particular
entre dois táxons. Esses elementos podem fazer parte da sua questão de
pesquisa, mas não são uma questão em si.

Quando você se fizer esta pergunta ou respondê-la para outras pessoas,
você deve ser capaz de expor claramente e em poucas palavras a questão
científica que você pretende investigar e qual a relevância dessa
questão para o seu campo de conhecimento.

\section{Use bem a estatística}

A estatística é o conhecimento instrumental mais importante hoje para
a biologia, além da língua inglesa. É a análise estatística que liga
nossos dados a hipóteses e teorias. Bons usuários sabem que há mais de
uma abordagem estatística e compreendem as suas diferenças
essenciais. Podem não conhecer todas a ponto de se sentirem seguros a
aplicá-las, mas dedicam-se a compreender criticamente e
operacionalmente pelo menos uma delas.

\textbf{Não é preciso ser estatístico(a) para usar bem a estatística
  na pesquisa científica. Mas é preciso conhecer a lógica das rotinas
  que vai usar}. Assim, basta que a estatística não seja para você uma
caixa preta onde entram dados e de onde saem resultados. Esse bom
entendimento operacional da estatística é um objetivo de longo prazo
que demanda muito estudo e, em muitos casos, o aprendizado de uma
linguagem computacional que o livre da camisa de força dos pacotes
estatísticos. Por isso, comece logo sua educação de pesquisador(a) que
faz uso consciente da estatística. Para ajudá-la(o), o PPGE- USP
oferece disciplinas básicas como \emph{Planejamento e Análise de Dados
  em Ecologia} \footnote{\url{http://labtrop.ib.usp.br/doku.php?id=cursos:planeco/}} e
\emph{Uso da Linguagem R para Análise de Dados em Ecologia}
\footnote{\url{http://ecor.ib.usp.br}}, além de outras
mais avançadas como \emph{Modelagem Estatística para Ecologia e
  Recursos Naturais} \footnote{\url{http://cmq.esalq.usp.br/BIE5781}}
ou ainda \emph{Tópicos Avançados em Ecologia}.

\section{Aprenda a expressar suas ideias e resultados para seus pares}

Durante sua carreira como pesquisador(a), você passará muito tempo
tentando explicar conceitos, hipóteses e resultados para seus
pares. \textbf{Uma das qualidades fundamentais de um bom pesquisador é
  a habilidade de se comunicar de forma clara, simples e direta}, seja
verbalmente ou por escrito. Entretanto, essa habilidade não surge de
forma miraculosa no momento em que você ingressa na
pós-graduação. Durante todo o seu período de permanência na
pós-graduação, você deve se expor constantemente a situações nas quais
você precisa falar em público. Em cada uma dessas oportunidades, você
estará exercitando sua capacidade de comunicação verbal. Aprenda com
seus erros e aproveite as apresentações de outros colegas e
profissionais mais experientes para incorporar aspectos positivos nas
suas próprias apresentações. Uma excelente oportunidade para começar a
se exercitar são os Seminários
EcoEncontros. (pág. \pageref{subsec:seminarios}).

Adicionalmente, é muito importante que você aprenda a se comunicar bem
de forma escrita. Esta é uma habilidade para a qual estudantes recebem
em geral pouco treinamento formal durante a graduação, o que pode
levar a dificuldades na pós-graduação para escrever projetos,
dissertações, teses ou artigos. Aproveite sua pós-graduação para
exercitar a redação de textos científicos. Temos a disciplina
\emph{Ecologia de Campo}
\footnote{\url{http://ecologia.ib.usp.br/curso/}} e outras em que
entre as habilidades exercitadas está o treinamento na lógica da
pesquisa e a comunicação científica.

\section{Exponha-se às críticas e comentários dos seus pares}


Atribui-se ao astrônomo Johannes Kepler a frase \emph{``prefiro a
  crítica mais dura de uma pessoa inteligente à aprovação irrefletida
  de uma multidão''}.  No meio acadêmico, devemos buscar as críticas
inteligentes às nossas ideias. Para isso expomos nosso resultados à
apreciação dos nossos pares em seminários, palestras, apresentações
orais em congressos e submissão de manuscritos. O desejo sincero de se
expor às críticas e comentários dos nossos pares, assim como \textbf{a
  maturidade de compreender que as críticas são direcionadas ao seu
  trabalho e não a você como indivíduo}, são passos importantes na sua
formação como pesquisador(a).

\section{Lembre-se que fazer ciência é uma atividade social}

Você não poderá fazer muito progresso como pesquisador(a) a menos que
você esteja disposto a procurar ajuda de outras pessoas e, em retorno,
oferecer ajuda àqueles que necessitam. \textbf{As maiores questões em
  ecologia demandam muito mais conhecimento teórico e habilidades
  técnicas do que qualquer pessoa possa adquirir durante toda a vida.}
Portanto, é impossível trabalhar de forma isolada e você deve estar
disposto(a) a colaborar com outras pessoas se deseja responder questões
maiores. Esta colaboração envolve não somente seus coautores, mas
também a discussão de seus projetos com colegas do Programa e de
outras universidades do Brasil e do exterior. Um dos grandes prazeres
da ciência é discutir ideias com pessoas com diferentes formações e
interesses.


\chapter[Dicas dos alunos]{Dicas dos seus colegas}

O ingresso em um novo programa de pós-graduação é um momento que nos
traz muita alegria e satisfação, mas vem repleto de novos desafios. O
nosso programa possui muitas oportunidades para novas(os) alunas(os)
interagirem com funcionárias(os), professoras(es) e alunas(os)
``veteranas(os)'', o que pode ajudar a esclarecer as dúvidas que possam
surgir nesse começo. No entanto, existem alguns detalhes no Programa e
na Universidade que podem facilitar nossa vida acadêmica e que não nos
ocorre perguntar. Por isso, as(os) alunas(os) criaram essa seção
no Manual para incluir dicas de coisas que aprenderam ao longo
da pós e que julgam valiosas para o ingressante.


\section{Sala da Pós-graduação}\label{sala-da-posgraduaucao}

A ``salinha'' possui entrada controlada por digital e, portanto, é
necessário cadastrar a sua com o técnico de informática (procure o
Luis). Essa ``salinha'' possui computadores de uso comum que podem
ser utilizados caso você não tenha um de uso pessoal ou não queira
levá-lo à USP. Nesses computadores, estão instalados alguns softwares
que utilizamos bastante na pós para os quais é necessário comprar a
licença. A sala também pode ser utilizada para reuniões entre os
alunos.

Além disso, a sala também possui uma impressora preto e branco, de uso
comum. Não há limites de impressão por pessoa e o papel e tonner (caso
estejam em falta) ficam disponíveis para os alunos na secretaria. É
permitido, inclusive, imprimir sua dissertação ou tese. É possível
também fazer impressões coloridas. Mas para isso, é necessário pedir
para o Luis e enviar para ele o documento a ser impresso e o número de
cópias. A nossa obrigação com a sala da Pós é evitar contaminar os
computadores com programas maliciosos (vírus), não instalar programas desnecessários e excluir
todos os arquivos que baixamos e que não iremos mais utilizar.


\section{Representação discente}\label{representacao-discente}

Os alunos da Ecologia têm um papel de extrema importância para o
funcionamento do nosso programa, pois a CCP leva em consideração a
nossa opinião e deixa a nosso encargo diversas atividades, como, por
exemplo, a distribuição de parte da verba PROEX.

A representação discente (RD) é composta por dois alunos, um titular e
um suplente, que participam da CCP e têm poder de voto, assim como os
docentes. Ao contrário dos outros membros da CCP, o RD tem seu mandato
válido somente por um ano, podendo ser reeleito caso não defenda antes
do prazo. O papel do RD é coordenar a comunicação entre discentes e
levar demandas para a CCP e aos docentes. Devido a essa importância
que temos, é essencial sempre contarmos com um representante discente
oficial para que ele possa ter voto válido na CCP.

O aluno que é RD possui um aprendizado diferenciado e ganha muito em
experiência em assuntos acadêmicos.

\section{CEPEUSP}\label{cepeusp}

Os trabalhos que desempenhamos no mestrado e doutorado podem ser
bastante prazerosos e, ao mesmo tempo, bastante estressantes. Nada
melhor para aliviar o estresse do que praticar alguma atividade
física. A comunidade USP dispõe de um Centro de Práticas esportivas
com diversas atividades gratuitas ou a preços bastante acessíveis.

Para saber mais sobre como utilizar esse serviço, basta acessar o site do
CEPEUSP\footnote{\url{http://www.cepe.usp.br/}}.

\section{Restaurantes Universitários}\label{Restaurantes Universitários}

No campus Butantã da Universidade de São Paulo, há quatro restaurantes
universitários: Central, da Física, da Prefeitura e da Química. São
servidas três refeições: café da manhã (somente no restaurante
Central), almoço e jantar. Para acessar os restaurantes, carregue seu
cartão USP em guichê localizado em frente ao restaurante Central ou
através do aplicativo Cardápio USP
({\url{http://www.app.usp.br/?page_id=70}}). Para mais informações
sobre horários e cardápios, acesse:
{\url{https://www.usp.br/coseas/COSEASHP/COSEAS2010_restaurantes.html}}.
